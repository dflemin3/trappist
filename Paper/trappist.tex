\pdfoutput=1
\documentclass[twocolumn]{aastex62}
%%\documentclass[]{emulateapj}

%Accepted/received/... %%

\received{xxx}
\revised{yyy}
\accepted{zzz}

%% Command to document which AAS Journal the manuscript was submitted to.
\submitjournal{AAS Journals}

%% Short title/authors

\shorttitle{LXUV History of TRAPPIST-1}
\shortauthors{Fleming et al.}

%% Begin document, title, packages %%
\usepackage{hyperref}
\usepackage{xspace}
\usepackage{graphicx}
\usepackage{amsmath}
\usepackage[caption=false]{subfig}

%% Custom commands
\def\mearth{{\rm\,M_\oplus}}
\def\rearth{{\rm\,R_\oplus}}
\def\msun{{\rm\,M_\odot}}
\def\rsun{{\rm\,R_\odot}}
\def\lsun{{\rm\,L_\odot}}
\def\gsim{~\rlap{$>$}{\lower 1.0ex\hbox{$\sim$}}}
\def\lsim{~\rlap{$<$}{\lower 1.0ex\hbox{$\sim$}}}

\newcommand{\xxx}[1]{{\textbf{#1}}}
\newcommand{\vplanet}[0]{\texttt{VPLanet}\xspace}
\newcommand{\emcee}[0]{\texttt{emcee}\xspace}
\newcommand{\approxposterior}[0]{\texttt{approxposterior}\xspace}
\newcommand{\eqtide}[0]{\texttt{EQTIDE}\xspace}
\newcommand{\stellar}[0]{\texttt{STELLAR}\xspace}
\newcommand{\kepler}[0]{\textit{Kepler}\xspace}
\newcommand{\jwst}[0]{\textit{JWST}\xspace}

%% Begin doc %%
\begin{document}

\title{On The XUV Luminosity Evolution of TRAPPIST-1}

%% AUTHORS %%

%%\correspondingauthor{David P. Fleming}
%%\email{dflemin3@uw.edu}

%%\author[0000-0001-9293-4043]{David P. Fleming}
\author{David P. Fleming}
\affil{Astronomy Department, University of Washington \\
Box 951580, Seattle, WA 98195}
\affil{NASA NExSS - Virtual Planetary Laboratory Lead Team, USA}

\author{Rory Barnes}
\affiliation{Astronomy Department, University of Washington \\
Box 951580, Seattle, WA 98195}
\affil{NASA Astrobiology Institute - Virtual Planetary Laboratory Lead Team, USA}

\author{Jake VanderPlas}
\affiliation{Google \\
601 N 34th St, Seattle, WA 98103}

\author{Rodrigo Luger}
\affil{NASA NExSS - Virtual Planetary Laboratory Lead Team, USA}
\affiliation{Center for Computational Astrophysics, Flatiron Institute \\
New York, NY 10010}

%% ABSTRACT %%

\begin{abstract}

The James Webb Space Telescope (JWST) is poised to detect and characterize the first terrestrial exoplanet atmosphere in the search for biosignatures \citep{Morley2017,Lustig2019}, but the correct interpretation of those observations is predicated on understanding the system's long-term evolution. Here, we derive probabilistic constraints for TRAPPIST-1's XUV luminosity ($L_{XUV}$) evolution using Markov Chain Monte Carlo (MCMC), accounting for observational uncertainties and correlations between parameters. We apply \approxposterior, a machine learning Python package for Bayesian inference, to compute accurate approximations of posterior distributions and show that this method obtains nearly identical results as traditional MCMC methods, but requires $800\times$ less computational time. We find that there is a $46.2\%$ chance that TRAPPIST-1 is still in the saturated phase today, likely significant atmospheric mass and water loss from its planets.

\end{abstract}

%% Keywords %%

\keywords{}

%% Intro %%

\section{Introduction} \label{sec:intro}

The James Webb Space Telescope (JWST) is poised to detect and characterize the first terrestrial exoplanet atmosphere in the search for biosignatures \citep{Morley2017,Lincowski2018,Lustig2019}, but the correct interpretation of those observations is predicated on understanding the system's long-term evolution, including processes that could impact the planet's habitability, like atmospheric escape \citep{Lammer2003,MurrayClay2009} and water loss \citep{Luger2015}. 

Talk about how XUV is important for atmospheric escape and water loss, e.g. its the major input, and knowing how it changes over time is critical for modeling these planets' atmospheres as a function of time. Then, explain that observations have elucidated how the activity and high-energy radiation of low-mass stars evolves as a function of time. Studies have probed the evolution of the ensemble, but not of individual stars to examine the evolutionary consequences on any planets they may harbor. TRAPPIST-1 is a well-studied nearby M dwarf that hosts 7 transiting exoplants that are prime targets of JWST transmission spectroscopy observations to detect and potentially characterize the first terrestrial exoplanet atmosphere (mention GTO program here). Our theoretical study is timely as we constrain the stellar and XUV evolution of TRAPPIST-1, leaving the water loss modeling, given our constraints, for future work. This study is not just required for TRAPPIST-1, but is required for any planet whose atmospheres we wish to characterize, so it must be computationally-reasonable.

- we constrain the stellar evolution and mass of TRAPPIST-1 using best priors and constraints currently available. we make our posterior distributions available for use with water loss simulations as the FXUV is a core input to such models, be it energy limited or recombination limited, maybe non-thermal escape models as well (look into that). our model and machine learning approach is all open source and documented on github, including a repo for the project itself
- TRAPPIST-1 conference stellar evolution stuff is 1st day, so email invited talks presenters and send them draft and thank them for their work which I use and cite in my own work - adam burgasser and jeff linsky
- from Peacock+2018: "Late-type M stars remain UV active for much longer than their early M counterparts, typically with more variability in the FUV than the NUV (Miles Shkolnik 2017)."

%From Rory's XRP: "...escape via “thermal” or “non-thermal” mechanisms.  The former canbe further divided into Jeans escape and hydrodynamic escape, which represent two end memberforms of atmospheric loss.  The former involves escape from above the exobase by particles withhigh thermal energies, the latter involves a collisional flow that originates near the planet’s XUVphotosphere.  Stellar XUV heating can play an important role in either case, but has been shown tobe especially important in hydrodynamic escape (e.g.,Murray-Clayet al., 2009).  The thermal massloss rate can be divided into energy-limited, diffusion-limited, and radiation-recombination-limitedregimes  in  which  one  or  a  few  parameters  implicitly  include  the  roles  of  atmospheric  structure,incident radiation, and photochemistry (e.g.,Watsonet al., 1981; Murray-Clayet al., 2009; Lugerand Barnes, 2015).  Non-thermal escape mechanisms are far more complicated and include processeslike  ablation,  ion  pick-up,  and  sputtering  (Airapetianet al.,  2019).   Internal  magnetic  fields  ofterrestrial planets may also be important,  as they can either deflect ionized stellar particles andprevent a collision with the atmosphere (decreasing atmospheric loss rates), or they may funnel high-energy particles toward the magnetic poles and heat the atmosphere there (increasing atmosphericmass loss rates) (Owen and Adams, 2014; Eganet al., 2019).  In general, thermal and non-thermalmechanisms can operate in tandem, or a planet can evolve through many of the cases describedabove.  Any model for the evolution of short-period exoplanets focused on atmospheric retentionmust account for all these heating, cooling, and escape mechanisms."

\citet{Newton2017} find that LHalpha, an activity indicator saturates as low Prot/Ro with the cutoff occuring at a similar Ro to other inidicators like Lx

-For low-mass stars in the saturated phase, $L_{X}$ remains a constant fraction of $L_{bol}$ and is observed to cluster around $\log_{10}(L_X/L_{bol}) \approx -3$ \citep{Pizzolato2003,Wright2011,Wright2018}. 

"Although the dynamo processes that drive stellar activity and XUV emission in fully-convective stars have been shown to evolve qualitatively similar to solar-type stars \citep{Wright2016,Wright2018} ..."

- define XUV, what it is, and why it's important

-Introduction about stellar activity for late type stars, what it is like and how late type stars evolve, and finally why it matters for exoplanet habitability, e.g. the XUV flux is a critical parameter for atmospheric escape and water loss, other loss mechanisms scale as FXUV to some power, like energy-limited and recombination-limited escape. In the intro, I explain what saturated and unsaturated phases correspond to physically, and under what conditions they are expected to occur. Argue how this qualitative behavior is observed for both partially and fully-convective stars, e.g. FGK and M dwarfs.

-The saturation ratio shows tentative evidence of increasing with decreasing stellar mass \citep{Wright2011,Jackson2012}, consistent with the observation that later type low-mass stars are more active \citep[e.g.][]{West2008}. 

-"Stellar XUV flux cannot be reliably determined as it is heavily absorbed by the ISM
even for closest stars. However, XUV fluxes are required input parameters for the characterization
of photoionization and photodissociation which are crucial for modeling exoplanetary atmospheric
dynamics and escape (Ribas+2005; Airapetian+2017a; Garcia-Sage+2017; Johnstone+2018; 
Lammer 2018; Glocer+2018). This effort should be combined with empirical and semi-empirical
methods which reconstruct XUV fluxes using FUV/NUV and radiative transfer modeling efforts
(Linsky+2013; France +2016; Peacock+2018; Richey-Yowell+2019)."

-"The XUV stellar spectrum is crucial for understanding the
exoplanetary atmospheric evolution and its impact on habitable worlds as it drives and regulates
atmospheric heating, mass loss and chemistry on Earth-like planets, and thus is critical to the
long-term stability of terrestrial atmospheres."

-"The stellar XUV emission can be reconstructed using empirical and theoretical
approaches. Empirical reconstructions have already provided valuable insights on the level of
ionizing radiation from F, K, G and M dwarfs (Cuntz \& Guinan 2016; France et al. 2016; 2018;
Loyd et al. 2016; Youngblood et al. 2016; 2017). "

-"Lighter species, such as hydrogen, tend to escape more easily through thermal escape, but
strong XUV fluxes can also stimulate the loss of light and heavy ions through ionospheric
outflow and exospheric pick-up by stellar winds (Glocer et al. 2012; Airapetian et al. 2017a;
Kislyakova et al. 2014; Dong et. al. 2017; Lammer et al. 2008; 2018)."

-\citet{Chassefiere1997} for Venusian water loss

We describe our model and statistical methods in $\S$~\ref{sec:model} and $\S$~\ref{sec:mcmc}. We present our results in $\S$~\ref{sec:results}, demonstrate the ability of new machine learning methods to accurately and efficiently reproduce our analysis in $\S$~\ref{sec:approx}, and discuss the implications of our results in $\S$~\ref{sec:discussion}.

\section{Model} \label{sec:model}

We simulate TRAPPIST-1's stellar evolution using the \stellar module in \vplanet \citep{Barnes2019} that performs a bicubic interpolation over mass and age of the \citet{Baraffe2015} stellar evolution tracks. The \citet{Baraffe2015} models, also employed by both \citet{Burgasser2017} and \citet{vanGrootel2018} to constrain TRAPPIST-1's stellar properties, were computed for solar metallicity stars and hence are suitable for TRAPPIST-1 whose [Fe/H] is consistent with solar \citep{Gillon2016}, although \citet{Burgasser2017} argue TRAPPIST-1 has a slightly super-solar metallicity based on isochrone modeling.

The X-ray luminosity, L$_{X}$, evolution of fully-convective stars follows the same broken power law model examined for partially-convective FGK stars \citep{Wright2016,Wright2018}. We assume TRAPPIST-1's L$_{XUV}$ evolution traces that of L$_{X}$ use the model of \citet{Ribas2005},
\begin{align}
\label{eqn:lxuv}
\frac{L_\mathrm{XUV}}{L_\mathrm{bol}} = \left\{
				\begin{array}{lcr}
					f_\mathrm{sat} &\ & t \leq t_\mathrm{sat} \\
					f_\mathrm{sat}\left(\frac{t}{t_\mathrm{sat}}\right)^{-\beta_\mathrm{XUV}} &\ & t > t_\mathrm{sat}
				\end{array}
				\right.,
\end{align}
where $f_{sat}$ is the constant ratio of stellar XUV to bolometric luminosity during the saturated phase, $t_{sat}$ is the duration of the saturated phase, and $\beta_{XUV}$ is the exponent that controls how steeply L$_{XUV}$ decays after saturation.

%% MCMC %%

\section{Markov Chain Monte Carlo} \label{sec:mcmc}

We use \texttt{emcee} \citep{ForemanMackey2013}, a Python implementation of the \citet{Goodman2010} affine-invariant Metropolis-Hastings Markov Chain Monte Carlo (MCMC) sampling algorithm, to infer posterior probability distributions for our model parameters. These parameters comprise the state vector
\begin{equation} \label{eqn:state}
    \textbf{x} = \{m_{\star}, f_{sat}, t_{sat}, \mathrm{age}, \beta_{XUV}\},
\end{equation}
where $m_{\star}$ and age are the stellar mass and age, respectively, and the other parameters are defined by Eqn.~\ref{eqn:lxuv}.  MCMC sampling allows us to derive probability distributions for our model parameters that are conditioned on observations and our understanding on the underlying physics, while accounting for correlations between parameters and their uncertainties.  

\subsection{Priors} \label{sec:mcmc:priors}

With only two data points to condition our analysis (TRAPPIST-1's observed $L_{bol}$ and $L_{XUV}$, see $\S$~\ref{sec:mcmc:like}), our prior probability distributions will strongly impact our results. We leverage observations of TRAPPIST-1 and late M dwarfs to assemble the best available empirical constraints to serve as priors for our MCMC analysis.  

Following \citet{vanGrootel2018}, we rely on TRAPPIST-1's luminosity and age to constrain its mass and therefore adopt a uniform prior of $m_{\star} \sim \mathcal{U}(0.07, 0.11)$. We use the empirical age constraint estimated by \citet{Burgasser2017}, age $\sim \mathcal{N}(7.6, 2.2^2)$ Gyr, as their thorough analysis considered both observations of TRAPPIST-1 and a host of empirical age indicators for ultracool dwarfs. We construct an empirical $f_{sat} = \log_{10}(L_{XUV}/L_{bol})$ distribution from the sample of fully-convective, saturated M dwarfs with observed $L_{X}$ from \citet{Wright2011}. For each star in the \citet{Wright2011} sample, we follow \citet{Wheatley2017} and estimate $L_{XUV}$ as a function of L$_{X}$ using Eqn.~(2) from \citet{Chadney2015}. We find that the distribution is well-approximated by a normal distribution, $f_{sat} \sim \mathcal{N}(-2.92, 0.26^2)$, and we adopt it as our prior.  

The duration of the saturated phase is estimated to be $t_{sat} \approx 100$ Myr for FGK stars \citep{Jackson2012}. Studies of stellar activity of late type stars as a function of stellar age, or its proxy, rotation period, indicate that the activity lifetime, and hence duration of the saturated phase, is likely longer for later-type stars \citep{Shkolnik2014,Wright2011,West2015,GonzalezAlvarez2019}, with fully-convective M dwarfs potentially remaining active for Gyrs, up to the ages of field stars \citep{West2008,Schneider2018}. TRAPPIST-1's high observed L$_{X}$ \citep{Wheatley2017}, short photometric rotation period \citep[3.3 d, ][]{Luger2017}, and low Rossby number \citep[Ro $\approx 0.01$, ][]{Roettenbacher2017} suggest that TRAPPIST-1 is still saturated today \citep{Pizzolato2003,Wright2011,Wright2018,Garraffo2017,GonzalezAlvarez2019}. Both \citet{Roettenbacher2017} and \citet{Morris2018} suggest that the photometrically-determined rotation period is inaccurate, with the latter study proposing that the 3.3 d period corresponds to a characteristic timescale for active regions on the stellar surface. TRAPPIST-1's $v \sin i = 6$ km s$^{-1}$ \citep{Barnes2014}, however, implies a rotation period of $\approx 1$ d for $i = 90^{\circ}$, providing evidence that TRAPPIST-1's rapid rotation is physical and consistent with saturation \citep{Wright2018}. Given these constraints, we adopt a broad uniform $t_{sat}$ prior distribution over $0.1 - 12$ Gyr.

In the unsaturated phase, $L_{X}$, and hence $L_{XUV}$, decay exponentially with powerlaw slope $\beta_{XUV}$ \citep{Ribas2005}. \citet{Jackson2012} find that $\beta_{XUV}$ does not significantly vary with stellar mass in their sample of FGK stars. Since \citet{Wright2016} found that the X-ray evolution of fully-convective stars is qualitatively similar to that of partially-convective FGK stars, we adopt the $\beta_{XUV}$ distribution of late K dwarfs from the \citet{Jackson2012} sample as our prior, $\beta_{XUV} \sim \mathcal{N}(-1.18, 0.31^2)$.

%\begin{deluxetable}{lcc}
%\tabletypesize{\small}
%\tablecaption{Prior Distributions \label{tab:priors}}
%\tablewidth{0pt}
%\tablehead{
%\colhead{Parameter [units]} & \colhead{Prior} & \colhead{Notes}
%}
%\startdata
%$m_\star$ [$M_{\odot}$] & $\mathcal{U}(0.07, 0.11)$ & -- \\  
%$f_{sat}$ & $\mathcal{N}(-2.92, 0.26^2)$ & \citet{Wright2011}  \\
%$t_{sat}$ [Gyr] & $\mathcal{U}(0.1, 12)$ & -- \\
%age [Gyr] & $\mathcal{N}(7.6, 2.2^2)$ & \citet{Burgasser2017} \\
%$\beta_{XUV}$ & $\mathcal{N}(-1.18, 0.31^2)$ & \citet{Jackson2012}
%\enddata \vspace*{0.1in}
%\end{deluxetable}

\subsection{Likelihood Function and Convergence} \label{sec:mcmc:like}

We further condition our analysis on TRAPPIST-1's observed bolometric luminosity, $L_{bol} = 5.22 \pm{0.19} \times 10^{-4} \ L_{\odot}$ \citep{vanGrootel2018}, and $L_{XUV}/L_{bol}$ \citep{Wheatley2017}. We convolve the \citet{vanGrootel2018} $L_{bol}$ measurement with the $L_{XUV}/L_{bol}$ constraints from \citet{Wheatley2017}, finding $L_{XUV} = 3.9 \pm{0.5} \times 10^{-7} \ L_{\odot}$.

For a given state vector \textbf{x}, we define our loglikelihood function as
\small
\begin{equation} \label{eqn:lnlike}
\begin{split}
    \ln \mathcal{L} \propto & -\frac{1}{2} \left[ \frac{(L_{bol} - L_{bol}(\textbf{x}))^2}{\sigma_{L_{bol}}^2} + \frac{(L_{XUV} - L_{XUV}(\textbf{x}))^2}{\sigma_{L_{XUV}}^2} \right] \\
    & + \ln \mathrm{Prior}(\textbf{x})
\end{split}
\end{equation}
\normalsize
where $L_{bol}$, $L_{XUV}$ and $L_{bol}(\textbf{x})$, $L_{XUV}(\textbf{x})$ are the observed values and \vplanet outputs given \textbf{x}, respectively, $\sigma_{L_{bol}}$ and $\sigma_{L_{XUV}}$ are the observational uncertainties, and $\ln \mathrm{Prior}(\textbf{x})$ is the prior probability. 

We run our MCMC with 100 parallel chains for 10,000 iterations, initializing each chain by randomly sampling each element of \textbf{x} from their respective prior distributions. Each step of the MCMC chain, \vplanet takes \textbf{x} as input and simulates TRAPPIST-1's evolution up to its age, predicting $L_{bol}$ and $L_{XUV}$ to evaluate the likelihood function. We discard the first 500 iterations as burn-in and assess the convergence of our MCMC chains by computing the integrated autocorrelation length and acceptance fraction for each chain. We find a mean acceptance fraction of 0.45 and a minimum and mean number of iterations per integrated autocorrelation length of 75 and 110, respectively, indicating that our chains have converged \citep{ForemanMackey2013}.

%% Results %%

\section{MCMC Results} \label{sec:results}

In Fig.~\ref{fig:corner}, we display the posterior probability distributions for our model parameters derived by our MCMC analysis, adopting the median values of the marginal distributions as our best-fit solutions and derive the lower and upper $1 \sigma$ uncertainties using the 16th and 84th percentiles, respectively. 

TRAPPIST-1 likely maintained a large $L_{XUV}$ throughout its lifetime as we find $f_{sat} = -3.05^{+0.24}_{-.10}$ and $t_{sat} = 6.85^{+3.43}_{-3.15}$ Gyr, consistent with observed $L_{XUV}/L_{bol}$ and long activity lifetimes of late M dwarfs \citep{West2008,Wright2018}. The long upper-tail in the marginalized $f_{sat}$ distribution arises from the combination of the degeneracy between $f_{sat}$ and $t_{sat}$ and from our strong empirical $f_{sat}$ prior that disfavors $f_{sat} \gsim -2.7$. To produce TRAPPIST-1's observed $L_{XUV}$, larger values of $f_{sat}$ require shorter $t_{sat}$, and hence an earlier dynamo transition to powerlaw $L_{XUV}$ decay during the unsaturated phase, and vice-versa. From the posterior distribution, we infer that there is a $42.6\%$ chance that TRAPPIST-1 is still in the saturated phase today, i.e. P$(t_{sat} \geq \mathrm{ age } | \mathrm{data}) \approx 0.426$. Our analysis strongly disfavors short saturation timescales, with only a $0.4\%$ chance that $t_{sat} \leq 1$ Gyr, the saturation timescale adopted by \citet{Luger2015} in their analysis of water loss from exoplanets orbiting in the habitable zone of late M dwarfs. 

\begin{figure*}[t]
\centering
	\includegraphics[width=0.75\textwidth]{../Analysis/Corner/trappist1Corner.pdf}
   \caption{Joint and marginal posterior probability distributions for the TRAPPIST-1 stellar parameters given in Eqn.~(\ref{eqn:state}) made using \texttt{corner} \citep{ForemanMackey2016}. The black vertical dashed lines on the marginalized distributions indicate the median values and lower and upper uncertainties from the 16th and 84th percentiles, respectively. From the posterior, we infer that there is a $42.6\%$ chance that TRAPPIST-1 is still in the saturated phase today, potentially driving significant water loss and atmospheric escape from its planets.}%
    \label{fig:corner}%
\end{figure*}

We constrain TRAPPIST-1's mass to $m_{\star} = 0.089 \pm{0.0006}$ M$_{\odot}$, in excellent agreement with \citet{vanGrootel2018}. Our marginalized age and $\beta_{XUV}$ posterior distributions reflect their prior distributions as for the former, $L_{bol}$ is not sufficient to further constrain TRAPPIST-1's age beyond our adopted prior since the luminosities of ultracool dwarfs do not significantly change during the main sequence \citep{Baraffe2015}. We do not constrain $\beta_{XUV}$ past out prior because our model prefers to exploit the degeneracy between $f_{sat}$ and $t_{sat}$ to match TRAPPIST-1's observed $L_{XUV}$ instead of varying the slope of the $L_{XUV}$ decay during the unsaturated phase. Furtheremore, TRAPPIST-1 could still be saturated, rendering $\beta_{XUV}$ unnecessary to model $L_{XUV}$. Age and $\beta_{XUV}$ weakly correlate with $f_{sat}$, both requiring a narrow spread of $f_{sat} \approx -3.05$ for young ages and steeper $\beta_{XUV}$. $\beta_{XUV}$ and $t_{sat}$ are uncorrelated, except at short $t_{sat}$ where steep $\beta_{XUV}$ are disfavored as this evolution would underpredict $L_{XUV}$.

\subsection{TRAPPIST-1's Evolutionary History}

Here we consider plausible stellar evolutionary histories for TRAPPIST-1 by simulating 100 samples from the posterior distribution using \vplanet. We plot the evolution of TRAPPIST-1's $L_{bol}$, $L_{XUV}$, and radius in Fig.~\ref{fig:evol} and compare our models to the measured values. 

\begin{figure*}[t]
	\includegraphics[width=\textwidth]{../Analysis/Evol/trappist1Evol.pdf}
   \caption{Plausible evolutionary histories of TRAPPIST-1's $L_{bol}$ (left), $L_{XUV}$ (center), and radius (right) using 100 samples drawn from the posterior distribution and simulated with \vplanet. In each panel, the blue shaded regions display the 1, 2, and 3 $\sigma$ uncertainties. The insets display the marginalized distributions (black) evaluated at the age of the system, with the blue dashed lines indicating the observed value and +/- 1 $\sigma$ uncertainties. The radius, $L_{bol}$, and $L_{XUV}$ constraints are adopted from \citet{vanGrootel2018} and \citet{Wheatley2017}, respectively.}%
    \label{fig:evol}%
\end{figure*}

TRAPPIST-1 remains saturated throughout its $1$ Gyr pre-main sequence, with both $L_{XUV}$ and $L_{bol}$ decreasing by a factor of ${\sim}40$ before stabilizing on the main sequence. TRAPPIST-1's radius likely shrank by roughly a factor of 4 along the pre-main sequence. We derive a present-day radius $R_{\star} = 0.11 \pm{0.0008} \ R_{\odot}$ from the posterior distribution that is ${\sim} 8\%$ smaller than the \citet{vanGrootel2018} constraint, $R_{\star} = 0.121 \pm {0.003} \ R_{\odot}$, that was computed from their inferred mass and TRAPPIST-1's density \citep{Delrez2018}. This difference arises from the likely underprediction of TRAPPIST-1's radius by the \citet{Baraffe2015} models, consistent with stellar evolution models often underestimating the radii of late M dwarfs \citep{Reid2005,Spada2013,Jackson2019}, and suggests that TRAPPIST-1 might have super-solar metallicity to account for its inflated radius \citep{Burgasser2017,vanGrootel2018}. If we instead follow \citet{vanGrootel2018} and compute the radius from our marginalized stellar mass posterior distribution and the \citet{Delrez2018} observed density, we obtain $R_{\star} = 0.12 \pm{0.002} \ R_{\odot}$, in agreement with \citet{vanGrootel2018}.

Since TRAPPIST-1 could still be saturated today, its planetary system has likely experienced a persistent extreme XUV environment, potentially driving substantial atmospheric escape and water loss \citep[][]{Luger2015,Bolmont2017,Bourrier2017a}. In Fig.~\ref{fig:fluxes}, we probe the distribution of XUV fluxes, $F_{XUV}$, derived from our posterior distributions for each TRAPPIST-1 planet when the system was 0.01, 0.1, and 1 Gyr old. We normalize these values by the $F_{XUV}$ received by Earth during the mean solar cycle \citep[$F_{XUV,\oplus} = 3.88$ erg s$^{-1}$cm$^{-2}$,][]{Ribas2005} and assume the planets remained near their current semi-major axes after migration in the natal protoplanetary halted \citep{Luger2017}. We infer that TRAPPIST-1b likely received extreme $F_{XUV}/F_{XUV, \oplus} \approx 10^4-10^5$ during the early pre-main sequence before decaying to the present-day $F_{XUV}/F_{XUV, \oplus} \approx 10^3$, consistent with estimates from \citet{Wheatley2017}. The likely habitable zone planets, e, f, and g, similarly experienced severe XUV fluxes ranging from $F_{XUV}/F_{XUV, \oplus} \approx {\sim} 10^2 - 10^{3.5}$ throughout the pre-main sequence. The extended upper-tail of the $F_{XUV}$ distributions correspond to the large $f_{sat}$ values permitted by the posterior distributions.

\begin{figure}[]
	\includegraphics[width=\columnwidth]{../Analysis/Fluxes/fluxes.pdf}
   \caption{$F_{XUV}/F_{XUV,\oplus}$ for each TRAPPIST-1 planet derived from the \vplanet evolutions of samples drawn from the posterior distribution when the system was 0.01, 0.1, and 1 Gyr old. The latter age corresponds to the approximate age at which TRAPPIST-1 entered the main sequence.}%
    \label{fig:fluxes}%
\end{figure}

%% approxposterior %%

\section{Inference Using \approxposterior} \label{sec:approx}

The methods presented in this work can be applied to any late-type star to constrain its $L_{XUV}$ history, given suitable priors and observational constraints. Our MCMC analysis, however, required 3,700 core hours on the University of Washington Hyak supercomputer. The main computational cost is incurred by running a $10$s \vplanet simulation each MCMC step to evaluate the likelihood, requiring ${\sim}1,000,000$ simulations in total for the full MCMC. Assuming similar convergence properties, repeating this analysis for even a modest sample of 30 stars would require~${\sim} 110,000$ core-hours, a non-trivial computational expense. Performing this analysis with more computationally-expensive models, perhaps ones that additionally interpolate stellar models over metallicity to fit for [Fe/H], would only exacerbate this issue.

% Extra
%If using a slower model than ours, perhaps one that models stellar evolution by interpolating tracks over mass, age, and metallicity to additionally constrain [Fe/H], or if testing alternate models of $L_{XUV}$ evolution for model comparisons, the computational expense will grow, exacerbating this issue.

We apply \approxposterior\footnote{\approxposterior is publicly available on \href{https://github.com/dflemin3/approxposterior}{GitHub}.}, an open source Python machine learning package \citep{FlemingVanderPlas2018}, to compute an accurate approximation to the true MCMC-derived posterior distribution for TRAPPIST-1's XUV evolution, while minimizing the computational cost. \approxposterior, an implementation of the ``Bayesian Active Learning for Posterior Estimation" (BAPE) algorithm of \citet{Kandasamy2015}, trains a Gaussian process \citep[GP, ][]{Rasmussen2006} surrogate for the likelihood evaluation, learning on the results of \vplanet simulations. The GP is then used within an MCMC sampling algorithm to quickly obtain the posterior distribution. Following \citet{Kandasamy2015}, \approxposterior iteratively improves the GP's predictive ability by identifying high-likelihood regions in parameter space where the GP predictions are uncertain. \approxposterior then evaluates \vplanet in those regions to supplement the training set, improving the GP's predictive ability in the relevant regions of parameter space, while minimizing the number of forward model evaluations required for suitable predictive accuracy. Similar techniques using a GP surrogate model have been shown to rapidly and accurately infer \citep[e.g.][]{Bird2019,Rogers2019,Takhtaganov2019} and reconstruct \citep{McClintock2019} Bayesian posterior distributions for computationally-expensive cosmology studies.

To model the covariance between points, we use a squared exponential kernel,
\begin{equation} \label{eqn:kernel}
k(x_i, x_j) = \exp \left( - \frac{(x_i - x_j)^2}{2l^2} \right),
\end{equation}
where $x_i$ and $x_j$ are two arbitrary points in parameter space and $l$ is hyperparameter that controls the scale length of the correlations. We assume correlations in each dimension have different scale lengths and fit for each $l$ by optimizing the GP's marginal likelihood of the training set data using the Nelder-Mead algorithm \citep{Nelder1965}, randomly restarting this optimization 25 times to mitigate the influence of local extrema. We run \approxposterior for 10 iterations, training the GP on an initial training set of 250 \vplanet simulations distributed over parameter space as a Latin hypercube to maximize coverage. Each iteration, \approxposterior selects 100 new training points, alternating between the \citet{Kandasamy2015} and \citet{Wang2017} selection criteria. \approxposterior runs \vplanet at each point for a total of 1,250 training samples. 

\begin{figure*}[t]
\centering
	\includegraphics[width=0.75\textwidth]{../Analysis/Approx/apCorner.pdf}
   \caption{Same format as Fig.~\ref{fig:corner}, but derived by \approxposterior.}%
    \label{fig:approx}%
\end{figure*}

\approxposterior can derive an accurate approximation to the true posterior distribution with relatively little computational expense. We use \approxposterior within \emcee to obtain the approximate posterior distribution, depicted in Fig.~\ref{fig:approx}, following the same procedure described in $\S$~\ref{sec:mcmc}. We find that the \approxposterior-derived MCMC chain converges in a similar time frame to our MCMC that runs \vplanet each likelihood evaluation (here referred to as the fiducial MCMC, shown in Fig.~\ref{fig:corner}), but uses significantly less computational resources. \approxposterior requires $800\times$ fewer \vplanet simulations to build its training set and only 11 core hours to complete, a factor of $336\times$ faster than our fiducial MCMC. The marginalized constraints derived by both methods, listed in Table~\ref{tab:constraints}, are remarkably similar with both the medians and widths of the $1 \sigma$ uncertainties in good agreement. Moreover as seen in Fig.~\ref{fig:approx}, \approxposterior recovers the same non-trivial correlations between model parameters seen in the fiducial MCMC posterior distribution. The posterior distribution derived by \approxposterior, however, is not an exact match as it underestimates the width of both the age and $\beta_{XUV}$ constraints by ${\sim}30\%$ and predicts that there is a $36.7\%$ chance that TRAPPIST-1 is still saturated today, $13.6\%$ smaller than the fiducial MCMC-derived value.

\begin{deluxetable}{lcc}
\caption{Parameter Constraints} \label{tab:constraints}
\tabletypesize{\small}
\tablewidth{0pt}
\tablehead{
\colhead{Parameter [units]} & \colhead{\vplanet MCMC} & \colhead{\approxposterior MCMC}
}
\startdata
$m_\star$ [$M_{\odot}$] & $0.089 \pm{0.0006}$ & $0.089 \pm{0.0006}$ \\  
$f_{sat}$ & $-3.05^{+0.24}_{-0.10}$ & $-3.03^{+0.19}_{-0.10}$,  \\
$t_{sat}$ [Gyr] & $6.85^{+3.43}_{-3.15}$ & $6.55^{+3.47}_{-2.54}$ \\
age [Gyr] & $7.44^{+2.04}_{-2.13}$ & $7.57^{+1.44}_{-1.46}$ \\
$\beta_{XUV}$ & $-1.16^{+0.31}_{-0.30}$ & $-1.16^{+0.22}_{-0.21}$
\enddata \vspace*{0.1in}
\tablecomments{Best fit values and uncertainties are derived using the medians, $16^{th}$, and $84^{th}$ percentiles from the marginalized posterior distributions, respectively.}
\end{deluxetable}

%% Discussion %%

\section{Discussion and Conclusions} \label{sec:discussion}

The TRAPPIST-1 planets are prime targets for atmospheric detection and characterization via JWST transmission spectroscopy observations (TRAPPIST-1 b is the target of an approved \jwst~GTO program, Proposal \#1279, for example). The correct interpretation of those observations is predicated on understanding the host star's long-term evolution as the XUV environments of close-in M and ultracool dwarf planets strongly influence atmospheric escape, water-loss, and atmospheric photochemistry, all processes that impact their atmospheric state and potential habitability \citep{Lammer2003,Ribas2005,MurrayClay2009,Luger2015,Airapetian2019}. Here, we used MCMC to derive probabilistic constraints for TRAPPIST-1's stellar and $L_{XUV}$ evolution to characterize its long-term dynamo evolution and the evolving XUV environment of its planetary system. From the posterior distribution, we inferred that TRAPPIST-1 likely maintained high $L_{XUV}/L_{bol} \approx 10^{-3}$ throughout its lifetime, with a $42.6\%$ chance that TRAPPIST-1 is still in the saturated regime today. Our results indicate that ultracool dwarfs can sustain large $L_{XUV}$ in the saturated regime for Gyrs, but additional observations of $L_{XUV}$ of ultracool dwarfs are required for confirmation. The TRAPPIST-1 planets likely experienced significant XUV fluxes during TRAPPIST-1's 1 Gyr pre-main sequence, potentially driving extreme atmospheric erosion and water loss \citep{Bolmont2017,Bourrier2017a}. We will use the constraints derived in this work to examine water loss in the TRAPPIST-1 system in a subsequent work (Fleming et al., in prep).

We demonstrated that deriving an approximation to the posterior distribution using the machine learning Python package, \approxposterior \citep{FlemingVanderPlas2018}, significantly reduces the computational expense, requiring $800\times$ fewer \vplanet simulations and a factor of $336\times$ less core hours than traditional MCMC approaches. This massive reduction in compute time makes scaling our analysis to a larger sample of late-type dwarfs feasible. The approximate posterior distributions derived by \approxposterior accurately reproduced the non-trivial parameter correlations uncovered by our fiducial MCMC analysis and yielded constraints that were in good agreement with those derived by our fiducial MCMC. Furthermore, \approxposterior's efficient adaptive GP-based method makes accurate approximate Bayesian inference with more computationally-intensive models possible \citep{Kandasamy2015}, permitting the generalization of this analysis to account for additional physical effects that impact the long-term evolution of potentially-habitable ultracool and M dwarf planets \citep[e.g. tidal heating,][]{Barnes2017}. 

%% ACKNOWLEDGEMENTS %%
\acknowledgments
This work was facilitated though the use of advanced computational, storage, and networking infrastructure provided by the Hyak supercomputer system and funded by the Student Technology Fund at the University of Washington. DPF was supported by NASA Headquarters under the NASA Earth and Space Science Fellowship Program - Grant 80NSSC17K0482.  RB acknowledges support from the NASA Astrobiology Institute's Virtual Planetary Laboratory under Cooperative Agreement number NNA13AA93A.

%% SOFTWARE %%
\software{\approxposterior: \citet{FlemingVanderPlas2018}, \texttt{corner}: \citet{ForemanMackey2016}, \texttt{emcee}: \citet{ForemanMackey2013}, \texttt{matplotlib}: \citet{Hunter2007}, \texttt{numpy}: \citet{vanderWalt2011}, \vplanet: \citet{Barnes2019}} 

%% BIBLIOGRAPHY %%

\bibliography{trappist}

% End of file
\end{document}